\documentclass[11pt]{beamer}

\usepackage{cedilleverbatim}
\usepackage{times}
\usepackage{stmaryrd}
\usepackage{MnSymbol}
\usepackage[normalem]{ulem} %for \sout

\usepackage{tikz}
\usepackage{graphicx}
%\usepackage{epstopdf}
\usepackage{color}
%\usepackage{tabularx}
\usepackage{pgflibraryarrows}
\usepackage{pgflibraryshapes}
\usepackage{pgfbaseimage}
\usepackage{listings}

\usetikzlibrary{decorations.text}

%\documentclass{beamer}
%\usepackage{beamerthemesplit}
%\usepackage{mathptmx}
%\usepackage{helvet}

%\usepackage{amsmath}
\usepackage{latexsym}
%\usepackage{amssymb}
\usepackage{proof}
\usepackage{verbatim} 
\usepackage{url}

\definecolor{navy}{RGB}{0,0,128}
\definecolor{dorange}{RGB}{255,100,0}
\definecolor{dgreen}{RGB}{0,75,0}
\definecolor{dbrown}{RGB}{125,30,0}
\definecolor{lbrown}{RGB}{245,150,120}
\definecolor{ored}{RGB}{255,150,70}
\definecolor{oredd}{RGB}{255,69,0}
\definecolor{lightred}{RGB}{255,70,70}
\definecolor{vlred}{RGB}{255,170,170}
\definecolor{cream}{RGB}{255,253,208}
\definecolor{lthistle}{RGB}{185,175,185}

\newcommand{\semeq}[1]{\langle\hspace{-.08cm}|#1|\hspace{-0.08cm}\rangle}
\newcommand{\rep}[1]{\ulcorner #1 \urcorner}
\newcommand{\tlift}[2]{{\uparrow}_{#1}\,(#2)}
\newcommand{\dstar}[0]{\smallstar}
\newcommand{\utp}[0]{\mathcal{U}}
\newcommand{\elcap}[0]{\cap}
\newcommand{\nutt}[4]{\nu\, #1\! :\! #2 \, |\, #3.\, #4}
\newcommand{\interp}[1]{\llbracket #1 \rrbracket} 
\newcommand{\abs}[4]{{#1}\, #2\! : \! #3.\, #4}

% unicode
\usepackage[utf8]{inputenc}
\DeclareUnicodeCharacter{2605}{\ensuremath{\star}}
\DeclareUnicodeCharacter{2081}{\ensuremath{_1}}
\DeclareUnicodeCharacter{2082}{\ensuremath{_2}}
\DeclareUnicodeCharacter{2228}{\ensuremath{\vee}}
\DeclareUnicodeCharacter{27E6}{\ensuremath{\llbracket}}
\DeclareUnicodeCharacter{27E7}{\ensuremath{\rrbracket}}
\DeclareUnicodeCharacter{228E}{\ensuremath{\uplus}}
\DeclareUnicodeCharacter{2113}{\ensuremath{\ell}}
\DeclareUnicodeCharacter{2294}{\ensuremath{\sqcup}}
\DeclareUnicodeCharacter{2192}{\ensuremath{\to}}
\DeclareUnicodeCharacter{2200}{\ensuremath{\forall}}
\DeclareUnicodeCharacter{22CE}{\ensuremath{\curlyvee}}
\DeclareUnicodeCharacter{2115}{\ensuremath{\mathbb{N}}}
\DeclareUnicodeCharacter{2238}{\ensuremath{\dotdiv}}
\DeclareUnicodeCharacter{2261}{\ensuremath{\equiv}}
\DeclareUnicodeCharacter{3BB}{\ensuremath{\lambda}}
\DeclareUnicodeCharacter{1D539}{\ensuremath{\mathbb{B}}}
\DeclareUnicodeCharacter{1D543}{\ensuremath{\mathbb{L}}}
\DeclareUnicodeCharacter{1D54A}{\ensuremath{\mathbb{S}}}
\DeclareUnicodeCharacter{1D54B}{\ensuremath{\mathbb{T}}}
\DeclareUnicodeCharacter{1D54D}{\ensuremath{\mathbb{V}}}
\DeclareUnicodeCharacter{D7}{\ensuremath{\times}}
\DeclareUnicodeCharacter{25C2}{\ensuremath{:}}

\newcommand{\myb}[0]{\ensuremath{\textcolor{blue}{\triangleright}}}

\newcommand{\fore}[0]{\ensuremath{F_\omega^{\textit{rec}}}}


\mode<presentation>
{
  %\usetheme{Warsaw}
  % or ...

%\usetheme{IowaCity}
\usetheme{Boston}
%\usetheme{Savannah}

%  \setbeamercovered{transparent}
  % or whatever (possibly just delete it)
}

\usepackage[english]{babel}
% or whatever

\usepackage{times}
\usepackage[T1]{fontenc}
% Or whatever. Note that the encoding and the font should match. If T1
% does not look nice, try deleting the line with the fontenc.



\date{\ }

\begin{document}


\setbeamercolor{normal text}{bg=white,fg=black}

\begin{frame}

\begin{center}
{\Large
  Introduction to Programming and \\
  Proving in Cedille }

\vspace{.2cm}

\includegraphics[width=3cm]{logo}

%\includegraphics[width=5cm]{oldcap}

\vspace{.6cm}

Chris Jenkins, Colin McDonald, Aaron Stump

{\small
Computer Science 

The University of Iowa

Iowa City, Iowa}

\end{center}
\end{frame}

\newcommand{\grun}[1]{\textcolor{dgreen}{\underline{#1}}}

\newenvironment{planslide}[1]{%
\begin{frame}
\frametitle{Plan for the tutorial}

\large 

\begin{tabular}{l c l }

\raisebox{-.25\height}{\includegraphics[width=.7cm]{logo}}? &\ & \textcolor{#1}{Motivation and background for Cedille} \\ \\


$\vdash \textit{\textcolor{red}{C}e\textcolor{red}{D}il\textcolor{red}{LE}}$ &\ & \textcolor{#1}{Syntax and semantics}\\ \\

\texttt{cedille} &\ & \textcolor{#1}{Tooling: emacs frontend $\leftrightarrow$ backend} \\ \\

\texttt{c d ll} &\ & \textcolor{#1}{Spine-local type inference} \\ \\ 

$\leadsto\ \texttt{cedille}_{\texttt{core}}$ & \ & \textcolor{#1}{Elaboration to Cedille Core} \\ \\

%\begin{tikzpicture}[
%  decoration={
%    reverse path,
%    text effects along path,
%    text={cedille cedille cedille cedille cedille cedille cedille cedille cedille cedille cedille cedille cedille cedille
%      cedille cedille cedille cedille cedille cedille cedille cedille cedille.},
%    text effects/.cd,
%      text along path,
%      character count=\i, character total=\n,
%      characters={scale=1-\i/\n}
%    }
%]
%\draw [decorate] (0,0) 
%    \foreach \i [evaluate={\r=(\i/2400)^2;}] in {0,7,...,2380}{ -- (\i:\r)}; 
%\end{tikzpicture}

\raisebox{-.8\height}{\includegraphics[width=2cm]{cedillespiral}} &\ & 

\textcolor{#1}{Future directions}


\end{tabular}

\end{frame}
}

\planslide{white}
\planslide{red}


\end{document}
