\documentclass{beamer}
\usetheme{Boadilla}

\usepackage[utf8]{inputenc}
\usepackage[greek,english]{babel}

\usepackage{subcaption}

\usepackage{proof}
\usepackage{amsmath}
\usepackage{amssymb}
\usepackage{amsthm}
\usepackage{syntax}
\usepackage{float}
\usepackage{moresize}
\usepackage{verbatim}
\usepackage{graphicx}

\usepackage{soul}

\input{../../share/slti-macros2}

\graphicspath{ { ./ }}

\newcommand{\lineheight}{\textcolor{white}{()\ensuremath{\forall}}}
\newcommand{\mklheight}[1]{\textcolor{white}{#1}}

\title{Spine-local Type Inference in Cedille}
% \subtitle{and Contextual Type-argument Inference}
\author[Jenkins, Stump]{Christopher Jenkins and Aaron Stump}
\institute[CS, U. Iowa]{Computer Science \\ University of Iowa}
\date{ICFP '18 (Cedille Tutorial)}

% Nice `orientation' slide
\AtBeginSection[]
{
  \begin{frame}
    \frametitle{Outline}
    \tableofcontents[currentsection]
  \end{frame}
}

\begin{document}
\beamertemplatenavigationsymbolsempty

\begin{frame}
  \titlepage
\end{frame}

\begin{frame}
  \frametitle{Outline}
  \tableofcontents
\end{frame}

\section{Background and Motivation}
\subsection{Local Type Inference}
\begin{frame}
  \frametitle{What is ``Local Type Inference''?}

  \begin{itemize}
  \item Uses two main techniques
    \begin{itemize}
    \item \textit{Bidirectional typing rules:}
     \uncover<2->{
      \[
        \begin{array}{rclcr}
          \text{Synthesis mode:}
          & \ 
          & \aabs{\lambda}{x}{Nat}{x} & \textcolor{blue}{\Uparrow} & \textcolor{blue}{Nat \to Nat}
          \\ \text{Checking mode:}
          & 
          & \abs{\lambda}{x}{x} & \textcolor{red}{\Downarrow} & \textcolor{red}{Nat \to Nat}
        \end{array}
      \]}
    \item \textit{Local type-argument inference:}

      \uncover<3->{\[
        \begin{array}{lcl}
          \text{Let } &  \ann{id}{\abs{\forall}{X}{X \to X}} &
          \\ \text{Type } & \fbox{\ensuremath{id\ 0}} & \Uparrow Nat
          \\ \text{Infer } & X\!=\!\textcolor{blue}{Nat} & \text{from } \textcolor{blue}{0}
        \end{array}
      \]}

    \uncover<3->{\fbox{Local} and \textcolor{blue}{Synthetic}}
    \end{itemize}
  \end{itemize}
\end{frame}

\begin{frame}
  \frametitle{Why use local type inference?}
  \begin{itemize}
  \item It is a method of \textit{partial} type inference
    \begin{itemize}
    \item \textit{Complete} type inference: no annotations \textbf{ever}

      (e.g. \textit{Damas-Hindley-Milner} and ML)
    \item Undecidable for System F (let alone Cedille!)
    \end{itemize}
    \pause
  \item It is user-friendly
    \begin{itemize}
    \item Infers many type annotations
    \item Predictable annotation requirements
    \item Better-quality error messages
    \end{itemize}
    \pause
  \item It is implementer-friendly
    \begin{itemize}
    \item Relatively simple implementation
    \item \textit{Extensible}: new features added without threatening decidability
    \end{itemize}
  \end{itemize}
\end{frame}

\begin{frame}
  \frametitle{Local Type Inference in Cedille}
  Cedille provides a way to interrogate the two core features of LTI
  
  \begin{itemize}
  \item \textit{Bidirectional Typechecking} gets special highlighting
    \uncover<2->{
      \begin{itemize}
      \item In navigation mode, type \texttt{C-h 3} to see \textit{bidirectional
          highlighting}.
      \end{itemize}
      }
  \item \textit{Type-argument Inference} gets a dedicated buffer

    \uncover<3->{
      \begin{itemize}
      \item In navigation mode, type \texttt{m} to see the \textit{meta-variables
          buffer.}
      \item \textbf{all} meta-variables present, \textbf{where} introduced,
        \textbf{what} solutions
      \end{itemize} }
  \end{itemize}

  \ \\ \ \\
  \uncover<4->{
  Cedille also has a \underline{novel} inference system: \textit{spine-local type inference}}
\end{frame}

\begin{frame}
  \frametitle{Limitations of other LTI Systems}
  Why a novel system? Other local type inference systems can sometimes still
  require ``silly'' type annotations...

  \[
    \begin{array}{lccl}
      \text{Assume}
      & \ann{mkpair &}{& \abs{\forall}{X\ Y}{\textcolor<3>{blue}{X} \to Y \to Pair\
        \cdot \alert<5>{X}\ \cdot Y}}
      \\ \text{Type}
      & mkpair\ \textcolor<3>{blue}{(\abs{\lambda}{x}{x})}\ 0
      & \uncover<2->{\textcolor<2-3>{blue}{\Uparrow} & ???}
      \\ \uncover<4->{\text{Type } & mkpair\ \alert<5>{(\abs{\lambda}{x}{x})}\ 0
      & \alert<4-5>{\Downarrow} & Pair \cdot (\alert<5>{Nat \to Nat}) \cdot Nat}
    \end{array}
  \]

  \begin{itemize}
    \pause
  \item We do not expect to locally \textcolor<2->{blue}{synthesize} a type 
    \pause \pause
  \item ... but we would expect to \alert<4->{check} it against a type
    \pause
    \begin{itemize}
    \item We could call this ``contextual'' type-argument inference.
    \end{itemize}
    \pause
  \item Unfortunately, not done in the two major published LTI systems
    \begin{itemize}
      \item Popular ``unofficial'' extension (used in e.g. Scala, Rust)
    \end{itemize}
  \end{itemize}
\end{frame}

\begin{frame}
  \frametitle{Limitations of other LTI Systems (cont.)}
  \begin{itemize}
  \item Usually uses ``fully-uncurried'' function applications
    {\Large \[f(t_1,..,t_n)\]}
    \begin{itemize}
    \item Maximize available info at a single application
    \end{itemize}
    \pause
  \item Usually without partial type application (``all-or-nothing'')
    {\Large \[f[T_1,...,T_m](t_1,...,t_n)\]}
  \end{itemize}    
\end{frame}

\subsection{Spine-local Type Inference}
\begin{frame}
  \frametitle{Our Contributions}
  \begin{itemize}
  \item Type inference for some expressions not typed by other variants of local
    type inference, by using \textit{contextual} type-argument inference

    \ \\ \centering
    \only<1>{\mklheight{{\LARGE $t_1^{\Uparrow\Downarrow}$}}}
    \only<2->{{\LARGE $f\ t_1^{\Uparrow}\
        t_2^{\only<2>{\Uparrow}\only<3->{\alert{\Downarrow}}}\ t_3^{\Downarrow}
        \uncover<3->{\alert{\Downarrow T}}$}}

    \raggedright
  \item Precise, specificational account of this technique
  \item Better support function currying and partial type applications by being ``spine-local.''

    \ \\ \centering
    \mklheight{{\LARGE \fbox{$f(,a)$}}}
    \only<4>{{\LARGE \fbox{$f[S,T,V](t_1, t_2, t_3)$}}}
    \only<5->{{\LARGE \fbox{$f \cdot\!S \textcolor<6>{white}{\cdot\!T \cdot\!V}\ t_1\ t_2\ \textcolor<6>{white}{t_3}$}}}
    \mklheight{{\LARGE \fbox{$f(a,)$}}}
  \end{itemize}
\end{frame}

\begin{frame}
  \frametitle{Room for Improvement}

  Type inference in Cedille will continue to be improved upon. Some things we
  want to address:

  \begin{itemize}
  \item Higher-order type arguments must be provided explicitly:

    ($\aabs{\forall}{F}{\star \to \star}$)

    \uncover<2->{
      \begin{itemize}
      \item \textbf{Comming soon:} second-order \textit{matching} is decidable
        and finitary
      \end{itemize}
    }
  \item Type arguments inferred \textit{only} between applications:

    \only<1-2>{$nil \Downarrow List\ \cdot\!Nat$}
    \only<3->{\st{$nil \Downarrow List\ \cdot\!Nat$}}
    \uncover<3->{
      Must write $nil\ \cdot Nat$
      \begin{itemize}
      \item<4-> \textbf{Coming soon:} polymorphic subsumption

        $\abs{\forall}{A}{List\cdot\!A}\ \subcov\ List \cdot\!Nat$
      \end{itemize}
    }
  \end{itemize}
\end{frame}

\section{Detailed Example}

\begin{frame}
  \frametitle{Terminology}

  \begin{itemize}[<+->]
  \item \textit{Application head}: variable or abstraction
    \[x, \ \  \abs{\Lambda}{X}{t},\ \  \abs{\lambda}{x}{t}\]
  \item \textit{Application spine}: head followed by seq. of term, type arguments
    \[
      \fbox{$x\textcolor{white}{t_1}$}\fbox{$t_1\ t_2\ t_3$} \text{ vs } (((x\ t_1)\ t_2)\ t_3)
    \]
  \item \textit{Applicand:} Term in the function position of an application
    \[t_1 \text{ in } t_1\ t_2\]
  \item \textit{Maximal application:} spine that is not an applicand
    \[
      \begin{array}{lc}
        \text{Not max} & \underline{x\ t_1\ t_2}\ t_3
        \\ \text{Max} & \underline{x\ t_1\ t_2\ t_3}
      \end{array}
    \]
  \end{itemize}
\end{frame}

\begin{frame}
  \frametitle{Example -- High Level Goals}
  \[
    \begin{array}{cccl}
      \text{Assume}
      & const & : & \abs{\forall}{X\ Y}{X \to Y \to X}
      \\ \text{Assume}
      & zero & : & Nat
      \\ \text{Type}
      & const\ (\abs{\lambda}{x}{x})\ zero & \Downarrow & Nat \to Nat
    \end{array}
  \]
  \begin{itemize}[<+->]
  \item Check the spine $const\ (\abs{\lambda}{x}{x})\ zero$ against
    $\alert<3->{Nat \to Nat}$
  \item We fill in missing types to ``elaborate'' to $const\ \cdot\alert<+->{(Nat \to Nat)}\ 
    \cdot \textcolor<.->{blue}{Nat}\ (\aabs{\lambda}{x}{\alert<.->{Nat}}{x})\ \textcolor<.->{blue}{zero}$

  \item<3-> $X$ inferred \alert<.->{contextually}, $Y$ \textcolor<.->{blue}{synthetically}

    \uncover<4->{In Cedille: $\fbox{$const\ (\abs{\lambda}{x}{x})\ zero$}$}
    \begin{itemize}
    \item<4-> $?X : \star\ \triangleleft Nat \to Nat$
    \item<4-> $?Y : \star\ = Nat$
    \end{itemize}
  \item<5-> And term argument $\abs{\lambda}{x}{x}$ \textit{checked} against
    \alert{$Nat \to Nat$}
  \end{itemize}
\end{frame}

\begin{frame}
  \frametitle{Spine Local}
  \[\fbox{$const\ (\abs{\lambda}{x}{x})\ zero$} \Downarrow \alert<3->{Nat \to Nat}\]

  \begin{itemize}
  \item \textbf{Big idea:} locality for type-argument inference is \textit{the
      spine}
    \begin{itemize}
    \item<2-> \fbox{cage meta-variables} here

      never ``escape'' the spine or ``descend'' into the arguments

    \item<3-> some meta-vars inferred \alert{contextually}

      Using the expected result type
    \end{itemize}
  \item<4-> \textbf{Consequence:} you know where to look when type-argument
    inference fails!
  \end{itemize}

  \centering
  \ \\ \ \\
  \only<1>{\includegraphics[scale=0.35]{../assets/spine-space}}
  \only<2>{\includegraphics[scale=0.35]{../assets/spine-local}}
  \only<3->{\includegraphics[scale=0.35]{../assets/spine-contextual}}
\end{frame}

\begin{frame}
  \frametitle{Example - Details}
  \[
    \begin{array}{cccl}
      \text{Assume}
      & const & : & \textcolor<2>{blue}{\abs{\forall}{X\ Y}{\textcolor<6>{blue}{X} \to \textcolor<8>{blue}{Y} \to \textcolor<3>{blue}{X}}}
      \\ \text{Assume}
      & zero & : & \textcolor<8>{blue}{Nat}
      \\ \text{Type}
      & \textcolor<2>{blue}{const}\ (\abs{\lambda}{x}{x})\ zero
              & \Downarrow & \alert<3>{Nat \to Nat}\uncover<4->{\ \alert<4>{=[Nat \to Nat/X]}X}
      \\ \uncover<3->{
      \text{Match}
      & \textcolor<3>{blue}{X} & \ozzmat{X} & \alert<3>{Nat \to Nat}
                                              }
      \\ \uncover<5->{
      \text{Type}
      & \abs{\lambda}{x}{x} & \uncover<6->{\Downarrow} & \uncover<6->{[\alert<6>{Nat \to Nat}/X]\textcolor<6>{blue}{X}}
                                                         }
      \\ \uncover<7->{
      \text{Type}
      & zero & \uncover<8->{\Uparrow} & \uncover<8->{[\textcolor<8>{blue}{Nat}/Y]\textcolor<8>{blue}{Y}}
      }
    \end{array}
  \]

  \ \\
  \only<2>{Synthesize type for head}
  \only<3>{Match head return with expected return type}
  \only<4>{Get a contextual \textit{instantiation}}
  \only<5-6>{Type first argument \uncover<6>{in \textit{checking mode}}}
  \only<7-8>{Type second argument \uncover<8>{in \textit{synthesis mode}}}
  \only<9>{Conclude the spine has the expected type!}
  \mklheight{$SF\forall$}
\end{frame}

\section{Annotation Requirements}

\begin{frame}
  \frametitle{Annotation Requirements: Abstractions (1/3)}

  Saw how type inference \textit{works}. Now -- where does it
  \textit{need help?}
  
  \begin{itemize}
  \item<2-> \textit{Term and type abstractions} $\abs{\lambda}{x}{t}$, $\abs{\Lambda}{X}{t}$

    Bare abstractions can only be checked; require type / kind annotations to
    synthesize
  \item<3-> \textbf{But:} when is a term's type checked? In a spine?
    \begin{itemize}
    \item<5-> when the \alert{checked type} of the spine
    \item<6-> and the \textcolor{blue}{synthesized type} of the head
    \item<7-> and any \textcolor{blue}{synthesized types} from earlier args
    \item<8-> tells us the \textit{complete} type to \alert{check}
    \end{itemize}
  \end{itemize}

  \uncover<4->{\Large
    \[f^{\uncover<6->{\textcolor{blue}{\Uparrow}}}\ t_1^{\uncover<7->{\textcolor{blue}{\Uparrow}}}\ t_2^{\uncover<8->{\alert{\Downarrow}}}\ t_3 \alert<5->{\Downarrow T}\]
  }
\end{frame}

\begin{frame}
  \frametitle{Annotation Requirements: Type Arguments (2/3)}

  Saw how type inference \textit{works}. Now -- where does it
  \textit{need help?}

  \begin{itemize}
  \item \textit{Type arguments} (FO only)

    Must provide type args explicitly when not informed by
    \begin{itemize}
    \item<2-> Checked type for spine: $const\ (\abs{\lambda}{x}{x})\ zero
      \Downarrow \alert{[Nat \to Nat/X]}X$
    \item<3-> Synthesized type for args: $zero \Uparrow \textcolor{blue}{[Nat/Y]}Y$
    \end{itemize}
  \item<4> \textbf{Example:} $\abs{\forall}{\textcolor{gray}{Y}\ X}{X \to X}$

    $\textcolor{gray}{Y}$ doesn't occur in result or arg position

    $\implies$ must be instantiated explicitly.
  \end{itemize}
\end{frame}

\begin{frame}
  \frametitle{Annotation Requirements: Applicability (3/3)}

  Saw how type inference \textit{works}. Now -- where does it
  \textit{need help?}

  \begin{itemize}
  \item<2-> \textit{Type of a function} in a term or type application

    must reveal resp. a type quantifier or arrow
    \[
      \begin{array}{ccc}
        f\ \cdot\!T & \implies & f : \abs{\forall}{X}{S}
        \\ f\ t & \implies & f : \abs{\forall}{\vars{X}}{S \to T}
      \end{array}
    \]
  \item<3-> \textbf{Example:} assume $absurd : \aabs{\forall}{X}{\star}{X}$
    \begin{itemize}
    \item<4-> Will \textbf{not} type $absurd\ zero$
    \item<4-> Will type $absurd \cdot\!(Nat \to Nat)\ zero$

      Or even $absurd \cdot\!(\abs{\forall}{X}{X \to X})\ zero$
    \end{itemize}

  \item<5-> \textbf{Advantage:} meta-variables are \textit{one-to-one} with
    quantified type variables in functions!

    It's always easy to understand why a meta-var was introduced!
  \end{itemize}
\end{frame}

\begin{frame}
  \frametitle{Thanks!}

  \begin{itemize}
  \item \texttt{language-overview/type-inference.ced}

    For practical examples

  \item Paper can be found on arXiv

    (To appear in proceedings of IFL 2018)
  \end{itemize}
\end{frame}

% \subsection{Soundness and Completeness}
% \subsection{Annotations Requirements}
% \subsection{Future Work}

\end{document}